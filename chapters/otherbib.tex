\clearpage
\appendix
\chapter{附錄一}
\label{appendix}

這裡展示了一些 \LaTeX{} 的用法,供參考用。

\vspace{2cm}

尤拉恆等式(Euler's identity)被理察·費曼(Richard Feynman)喻爲「最美的公式」:
\begin{equation}
    e^{i \pi} + 1 = 0 \label{euler-identity}
\end{equation}

\vspace{2cm}

PID控制器可以表示為式\eqref{pid},其中$e$為誤差,定義為目標設定值減去實際回授值:$e=V_{exp}-V_{act}$。

\begin{equation}
    u(t) = \underbrace{K_p e(t)}_{\rm P-term}
         + \underbrace{K_i \int_{0}^{t} e(\tau)\mathrm{d}\tau}_{\rm I-term}
         + \underbrace{K_d \frac{\mathrm{d}}{\mathrm{d}t} e(t)}_{\rm D-term}
    \label{pid}
\end{equation}

\begin{align}
^{base}\mathbf{T}_{end} &= \; ^0\mathbf{T}_2 = \; ^0\mathbf{T}_1 \; ^1\mathbf{T}_2 \\
&=
\begin{bmatrix}
\cos \theta_1&0&\sin \theta_1&0\\
\sin \theta_1&0&-\cos \theta_1&0\\
0&1&0&0\\
0&0&0&1
\end{bmatrix}
\begin{bmatrix}
\cos \theta_2&-\sin \theta_2&0&l_1\cos \theta_2\\
\sin \theta_2&\cos \theta_2&0&l_1\sin \theta_2\\
0&0&1&d_2\\
0&0&0&1
\end{bmatrix}
\end{align}

上下標:$^1_2x^3_4$
